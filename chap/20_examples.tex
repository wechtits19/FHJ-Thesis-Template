\chapter{Some usage examples}
\label{ch:usage-examples}

\section{Section}
\label{sec:section}
Blablabla

\subsection{Subsection}
\label{subsec:subsection}
Sub-blalba

\subsubsection{Bitmap image}

\begin{figure}[!ht]
    \centering
    \includegraphics[width=0.5\textwidth, interpolate=false]{res/magnified_bitmap}
    \caption{Magnified raster graphic}
    \label{fig:Bitmap}
\end{figure}

\subsubsection{SVG image}
Die Nutzung von SVG-Grafiken funktioniert gleich wie die von \glqq normalen\grqq Bildern, aber es muss \textbf{inkscape} installiert sein.

\begin{figure}[!ht]
    \centering
    \includesvg[width=0.5\textwidth, inkscapearea=page]{res/magnified_vector}
    \caption{Magnified vector graphic}
    \label{fig:Vector}
\end{figure}

\subsubsection{Code}

Direkt hier:

\begin{minted}{python}
from scipy.signal import argrelmax

# Get curvature extrema
peaks = []
dips = []

for contour in contours_with_curvatures:
    smoothed_curvature = smooth(contour[1])[5:-5]
    peaks.append(argrelmax(smoothed_curvature, mode='wrap'))
    dips.append(argrelmax(-smoothed_curvature, mode='wrap'))

peaks = np.asarray(peaks, dtype=object)
dips = np.asarray(dips, dtype=object)
\end{minted}

\hspace{1cm}

Von einem File:

\inputminted[firstline=10, lastline=22]{python}{code/filterPeaksAndDips.txt}

Nutzung von Abkürzungen wie \ac{IDE} und eines Zitats~\cite{Selinger.2003}.
Und erneute Nutzung von Abkürzungen wie \ac{IDE}.